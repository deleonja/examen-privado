\documentclass[11pt,dvipsnames]{report} % {{{
\usepackage[spanish]{babel}
\usepackage[utf8]{inputenc}
\usepackage[T1]{fontenc}
\usepackage{makeidx}
\usepackage{graphicx}
\usepackage{subfig}
\usepackage{amsmath}
\usepackage{amsfonts}
\usepackage{amssymb}
\usepackage{authblk} % para la manipulación de autores y afiliación
\usepackage[pdftex]{hyperref}
\usepackage{multirow}
\usepackage{multicol}
\usepackage{float}
\usepackage{xcolor}
\usepackage{booktabs}
\usepackage{colortbl}
\usepackage{bbold}
\usepackage{physics}
\usepackage{mathtools}
\usepackage{dsfont}
\usepackage{tensor}

% Theorems, proofs, etc
\usepackage{amsthm}



\usepackage{fancybox}
\usepackage{colortbl}
\usepackage{amsbsy}
\usepackage[draft,inline,nomargin]{fixme} \fxsetup{theme=color}

%This defines my comments
\definecolor{mycolor}{RGB}{255,50,0}
\FXRegisterAuthor{ja}{aja}{\color{mycolor}JA}


\usepackage[]{lineno} \linenumbers
\setlength\linenumbersep{3pt}

\newcommand{\fref}[1]{fig.~\ref{#1}}  \newcommand{\tref}[1]{table~\ref{#1}}
\newcommand{\Fref}[1]{Fig.~\ref{#1}}  \newcommand{\Tref}[1]{Table~\ref{#1}}

\usepackage{hyperref}
%\usepackage{commath}
\decimalpoint
\renewcommand{\tablename}{Tabla}
\oddsidemargin 0in
\textwidth 6.5in
\topmargin -0.5in
\textheight 8.5in

\newcommand{\psii}{\psi_i}
\newcommand{\Pk}[1]{\ket{\psi_{#1} }}
\newcommand{\Pb}[1]{\bra{\psi_{#1} }}
\newcommand{\pk}{\ket{\psi}}
\newcommand{\M}{\mathcal{M}^{(N)}}
\newcommand{\E}{\mathcal{E}}
\newcommand{\Erho}{\mathcal{E}(\rho)}
\newcommand{\1}{\mathds{1}}
\newcommand{\ten}{\otimes}
\newcommand{\h}[1]{\colorbox{Yellow}{#1}}
\newcommand{\hi}{\mathcal{H}}
\newcommand{\txt}[1]{\text{#1}}
\newcommand{\here}{\h{\hspace{15cm}} }
\newcommand{\rhoi}{\dyad{\psii}{\psii}}
\newcommand{\ind}[2]{{{}^{#1}_{#2}}}
\newcommand{\QH}{\abs{Q_H}}
\newcommand{\QL}{\abs{Q_L}}
\newcommand{\W}{\abs{W}}

\def\dbar{{\mathchar'26\mkern-12mu d}}


% Para que funcione mejor la numeración {{{
% https://tex.stackexchange.com/questions/43648/why-doesnt-lineno-number-a-paragraph-when-it-is-followed-by-an-align-equation
\newcommand*\patchAmsMathEnvironmentForLineno[1]{%
  \expandafter\let\csname old#1\expandafter\endcsname\csname #1\endcsname
  \expandafter\let\csname oldend#1\expandafter\endcsname\csname end#1\endcsname
  \renewenvironment{#1}%
     {\linenomath\csname old#1\endcsname}%
     {\csname oldend#1\endcsname\endlinenomath}}% 
\newcommand*\patchBothAmsMathEnvironmentsForLineno[1]{%
  \patchAmsMathEnvironmentForLineno{#1}%
  \patchAmsMathEnvironmentForLineno{#1*}}%
\AtBeginDocument{%
\patchBothAmsMathEnvironmentsForLineno{equation}%
\patchBothAmsMathEnvironmentsForLineno{align}%
\patchBothAmsMathEnvironmentsForLineno{flalign}%
\patchBothAmsMathEnvironmentsForLineno{alignat}%
\patchBothAmsMathEnvironmentsForLineno{gather}%
\patchBothAmsMathEnvironmentsForLineno{multline}%
}
% }}}
% }}}
\title{Notas de estudio para Examen Privado\\
Licenciatura en Física}
\author{J.A. de León}

\newtheorem{definition}{Definición}[section]

\newtheorem{teorema}{Teorema}[section]

\newtheorem{property}{Propiedad}[section]

\begin{document}
\maketitle

\chapter{Termodinámica}
\section{Trabajo}
\begin{enumerate}
\item Durante un \textbf{proceso cuasiestático} un sistema atraviesa 
		en todo momento 
	  estados infinitesimalmente cercanos al equilibrio termodinámico y, 
	  por consiguiente, la dinámica de los estados puede ser descrita
	  en términos de coordenadas termodinámicas $\Rightarrow$ existe 
	  una ecuación de estado que describe estos estados 
\item Equilibrio termodinámico: eq. mecánico $+$ eq. térmico $+$ 
 	  eq. químico
\item Diferencial de trabajo de un sistema de un gas en un pistón:
\begin{equation}
\dbar W= PAdx = -PdV,
\end{equation}
el signo negativo asegura que en una compresión $(dV<0)$ $\dbar W>0$ 
(trabajo sobre el sistema, i.e. el sistema gana energía). 
\item El trabajo realizado sobre un sistema hidrostático depende de 
la trayectoria que se siga en el diagrama de estados (a diferencia
del trabajo realizado por la fuerza gravitacional, por ej). 
\item El trabajo relizado sobre un gas, definido por
\begin{align*}
W=-\int_{V_i}^{V_f} PdV, 
\end{align*}
se puede integrar siempre y cuando se conozca una función 
de estado.
\item ¡¡¡Calor $=$ energía!!!
\end{enumerate}

\section{1era ley de la Termodinámica}
\begin{itemize}
\item Una forma \textbf{restringida} de enunciar la primera ley 
de la termodinámica es:

Si un sistema cerrado es causado a cambiar de estado de un estado 
inicial a un estado final únicamente por procesos adiabáticos, entonces
el trabajo realizado sobre el sistema es el mismo para todas las trayectorias
adiabáticas que conectan a ambos estados. 

Es restringida porque no se está diciendo qué ocurre en 
aquellos procesos no adiabáticos. Es decir, en procesos
en los que ocurre intercambio de calor. 
\item Se sigue de esta forma restringida de la primera ley de la 
Termodinámica que existe una función que depende únicamente
de las coordenadas termodinámicas cuya diferencia entre el valor
final e inicial es igual al trabajo adiabático que se realiza al
sistema pasar del estado inicial al estado final. Esta función 
es la función de \textbf{energía interna} $U$: 
\begin{align}
W_{i\to f}\text{(adiabático)}=U_f-U_i,
\end{align}
esta diferencia da el aumento en energía interna del sistema. 
\item  \textbf{1era. ley de la Termodinámica} (formulación 
matemática): 
Cuando un sistema cerrado y su entorno se encuentran a distintas
temperatura y se realiza trabajo diatérmico [?] sobre el sistema,
entonces la energía transferida por otros procesos no mecánicos, 
igual a la diferencia entre el cambio de energía interna y el 
trabajo diatérmano se llama calor $Q$:
\begin{align}
  \Delta U = Q + W,
  \label{eq:1st-law-thermodynamics}
\end{align}
donde $Q>0$ cuando el calor entra al sistema y $Q<0$ cundo
abandona el sistema. Tres ideas están plasmadas en 
\eqref{eq:1st-law-thermodynamics}: (1) la existencia de una 
función de energía interna, (2) el principio de conservación de 
la energía, y (3) la definición de trabajo como energía en tránsito
en virtud de una diferencia de temperatura.
\item \textbf{Forma diferencial de la 1era. ley de la Termodinámica}:
para un sistema termodinámico que atraviesa cambios infinitesimales 
en sus variables termodinámicas la primera ley se formula como
\begin{align}
  dU = \dbar Q + \dbar W.
  \label{eq:1st-law-thermodynamics_diffForm}
\end{align}
\item \textbf{Capacidad calorífica}: 
\begin{align}
  C=\dv{Q}{T} \qty[\frac{K}{J}]
  \label{eq:heat-capacity}.
\end{align}
\item El caso de un gas:

La ecuación \eqref{eq:1st-law-thermodynamics_diffForm} toma 
la forma 
\begin{align}
   dU =\dbar Q-PdV,
   \label{eq:1st-law-gas}
 \end{align} 
donde $U$ es una función de algún par $P,V,T$ (la ecuación relaciona 
a un par con la tercera). Tomando $U=U(T,V)$ se tiene 
\begin{align*}
dU=\qty(\pdv{U}{T	})_VdT+\qty(\pdv{U}{V})_TdV,
\end{align*}
y sustityendo en \eqref{eq:1st-law-gas}
\begin{align*}
\dbar Q &=\qty(\pdv{U}{T	})_VdT+\qty[\qty(\pdv{U}{V})_T+P]dV\\
\dv{Q}{T} &= \qty(\pdv{U}{T	})_V
+\qty[\qty(\pdv{U}{V})_T+P]\dv{V}{T}.
\end{align*}
\item Flujo cuasiestático de calor: cuando a lo largo de un sistema 
existe un gradiente de temperatura y el calor fluye de manera
cuasiestática (misma definición) podemos calcular el 
calor que se absorbe durante el proceso utilizando la ecuación 
\eqref{eq:heat-capacity} de la capacidad calorífica.
\item El transporte de energía en un sistema entre elementos
de volumen vecinos en virtud de una diferencia de temperatura se 
conoce como conducción de calor. Los experimentos muestran que 
\begin{align*}
\frac{Q}{t}\propto A\frac{\Delta T}{\Delta x},
\end{align*}
con $A$ el área transversal al flujo de calor. Por consiguiente
\begin{align}
\dv{Q}{t	} = -KA\dv{T}{x},
\end{align}
con $K$ la conductividad térmica. El signo menos es para asegurar
que la dirección del flujo de calor sea en la dirección positiva de $x$.
\item \textbf{Radiación térmica}: radiación en virtud de su temperatura.
\item Exitancia radiante $\mathcal{R}$: potencia irradiada total por 
unidad de área. Emisividad total $\epsilon$: fracción de la potencia
irradiada total que es emitida como radiación térmica.
\item \textbf{Ley de Stefan-Boltzmann}: un cuerpo negro es una
sustancia ideal que es capaz de absorber toda la luz incidente 
sobre ella y reemitirla netamente como radiación térmica. La
radiación de un cuerpo como este está dada por la la siguiente ley:
\begin{align}
  \mathcal{R}=\mathcal{R}(T)=\sigma T^4,
  \label{eq:stefan-boltzmann}
\end{align}
donde $\sigma$ es la constante de Stefan-Boltzmann.
\end{itemize}

\section{Gas ideal}
\begin{enumerate}
\item \textbf{Expansión libre adiabática}: gas en expansión en el 
cual no se hace trabajo ni se transfiere calor $\Rightarrow$ $\Delta U=0$
durante la expansión libre.
\item Al escribir explícitamente $dU$ asumiendo que $U=U(T,V)$ ó 
$U=U(T,P)$ y al considerar una expansión libre adiabática $(dU=0)$
se concluye que $U=U(T)$ para un gas ideal. 
\item Un \textbf{gas ideal} satisface
\begin{align}
  PV&=nRT, & \qty(\pdv{U}{P})_T&=0,
  \label{eq:ideal-gas}
\end{align}
la primera condición es la ecuación de estado del gas ideal y 
la segunda es para asegurar que en una expansión libre
$dU=0$.
\item Retomando la ecuación \eqref{eq:1st-law-gas}:
\begin{align*}
\dbar Q=dU+PdV,
\end{align*}
cuando $dV=0$
\begin{align*}
\qty(\dv{Q}{T})_T=C_V=\qty(\pdv{U}{T})_V.
\end{align*}
En el caso del gas ideal $U=U(T)$, por consiguiente
\begin{align*}
C_V=\dv{U}{T},
\end{align*}
y 
\begin{align*}
\dbar Q = C_VdT+PdV,
\end{align*}
y sacando un diferencial de la ecuación de estado del gas ideal
\begin{align*}
PdV+VdP=nRdT.
\end{align*}
Sustiyendo arriba
\begin{align*}
\dbar Q &= C_VdT + nRdT - VdP\\
\dv{Q}{T}&=\qty(C_V+nR)-V\dv{P}{T},
\end{align*}
cuando $dP=0$:
\begin{align}
  \qty(\dv{Q}{T})_P=C_P=C_V+nR.
  \label{eq:C_P-idealGas}
\end{align}
\end{enumerate}






































\bibliographystyle{abbrv}
\bibliography{references}
\end{document}


