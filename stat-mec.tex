\chapter{Mecánica Estadística}


\section{Fundamentos estadísticos de la Termodinámica}
\begin{itemize}
	\item La especificación de los valores de los parámetros $N$, $V$ y $E$
	define a un \textbf{macroestado}.
	
	\item Los \textbf{microestados} se identifican como las posibles 
	configuraciones de un sistema para dar lugar al mismo valor 
	de energía $E$ del sistema completo, $\Omega=\Omega(N,V,E).$
	
	\item \textbf{Postulado de la igualdad de probabilidad apriori:} el sistema
	tiene igualdad de probabilidad de estar en cualquiera de los posibles
	microestados.
	
	\item Al colocar dos sistemas termodinámicos en contacto se supone
	que estos llegarán al equilibrio cuando se el valor de la energía
	del sistema (\janote{del sistema?}) maximice el número 
	de microestados en los que se puede encontrar el sistema.
	El macroestado con la mayor cantidad de microestados 
	se dice que es el más probable.
\end{itemize}