\documentclass[11pt,dvipsnames]{report} % {{{
\usepackage[spanish]{babel}
\usepackage[utf8]{inputenc}
\usepackage[T1]{fontenc}
\usepackage{makeidx}
\usepackage{graphicx}
\usepackage{subfig}
\usepackage{amsmath}
\usepackage{amsfonts}
\usepackage{amssymb}
\usepackage{authblk} % para la manipulación de autores y afiliación
\usepackage[pdftex]{hyperref}
\usepackage{multirow}
\usepackage{multicol}
\usepackage{float}
\usepackage{xcolor}
\usepackage{booktabs}
\usepackage{colortbl}
\usepackage{bbold}
\usepackage{physics}
\usepackage{mathtools}
\usepackage{dsfont}

% Theorems, proofs, etc
\usepackage{amsthm}

\definecolor{mycolor}{RGB}{255,50,0}

\usepackage{fancybox}
\usepackage{colortbl}
\usepackage{amsbsy}
\usepackage[draft,inline,nomargin]{fixme} \fxsetup{theme=color}
\FXRegisterAuthor{cp}{acp}{\color{blue}CP}
\FXRegisterAuthor{ja}{aja}{\color{mycolor}JA}
\FXRegisterAuthor{cc}{acc}{\color{Purple}CC}


\usepackage[]{lineno}  \linenumbers
\setlength\linenumbersep{3pt}

\newcommand{\fref}[1]{fig.~\ref{#1}}  \newcommand{\tref}[1]{table~\ref{#1}}
\newcommand{\Fref}[1]{Fig.~\ref{#1}}  \newcommand{\Tref}[1]{Table~\ref{#1}}

%\usepackage{hyperref}
%\usepackage{commath}
\decimalpoint
\renewcommand{\tablename}{Tabla}
\oddsidemargin 0in
\textwidth 6.5in
\topmargin -0.5in
\textheight 8.5in

\newcommand{\psii}{\psi_i}
\newcommand{\Pk}[1]{\ket{\psi_{#1} }}
\newcommand{\Pb}[1]{\bra{\psi_{#1} }}
\newcommand{\pk}{\ket{\psi}}
\newcommand{\M}{\mathcal{M}^{(N)}}
\newcommand{\E}{\mathcal{E}}
\newcommand{\Erho}{\mathcal{E}(\rho)}
\newcommand{\1}{\mathds{1}}
\newcommand{\ten}{\otimes}
\newcommand{\h}[1]{\colorbox{Yellow}{#1}}
\newcommand{\hi}{\mathcal{H}}
\newcommand{\txt}[1]{\text{#1}}
\newcommand{\here}{\h{\hspace{15cm}} }
\newcommand{\rhoi}{\dyad{\psii}{\psii}}
\newcommand{\ind}[2]{{{}^{#1}_{#2}}}

\def\dbar{{\mathchar'26\mkern-12mu d}}


% Para que funcione mejor la numeración {{{
% https://tex.stackexchange.com/questions/43648/why-doesnt-lineno-number-a-paragraph-when-it-is-followed-by-an-align-equation
\newcommand*\patchAmsMathEnvironmentForLineno[1]{%
  \expandafter\let\csname old#1\expandafter\endcsname\csname #1\endcsname
  \expandafter\let\csname oldend#1\expandafter\endcsname\csname end#1\endcsname
  \renewenvironment{#1}%
     {\linenomath\csname old#1\endcsname}%
     {\csname oldend#1\endcsname\endlinenomath}}% 
\newcommand*\patchBothAmsMathEnvironmentsForLineno[1]{%
  \patchAmsMathEnvironmentForLineno{#1}%
  \patchAmsMathEnvironmentForLineno{#1*}}%
\AtBeginDocument{%
\patchBothAmsMathEnvironmentsForLineno{equation}%
\patchBothAmsMathEnvironmentsForLineno{align}%
\patchBothAmsMathEnvironmentsForLineno{flalign}%
\patchBothAmsMathEnvironmentsForLineno{alignat}%
\patchBothAmsMathEnvironmentsForLineno{gather}%
\patchBothAmsMathEnvironmentsForLineno{multline}%
}
% }}}


% }}}
\begin{document}
% Titulo, abstract y esas vainas {{{
\begin{titlepage} % Suppresses displaying the page number on the title page and the subsequent page counts as page 1                                  
\newcommand{\HRule}{\rule{\linewidth}{0.6mm}} % Defines a new command for horizontal lines, change thickness here                             

\center % Centre everything on the page                                                                                                       

%------------------------------------------------                                                                                             
%       Title                                                                                                                                 
%------------------------------------------------                                                                                             

\HRule\\[0.4cm]

% Title of your document                                                 
{\LARGE\bfseries Mapeos proyectivos en sistemas de varios qubits}\\[0.4cm] 

\HRule\\[2cm]

%------------------------------------------------                                                                                             
%       Author(s)                                                                                                                             
%------------------------------------------------                                                                                             


\Large{\textbf{José Alfredo de León Garrido}}\\[2cm]

%------------------------------------------------                                                                                             
%       Headings                                                                                                                              
%------------------------------------------------                                                                                             

\textsc{\Large Universidad de San Carlos de Guatemala\\
		Escuela de Ciencias Físicas y Matemáticas\\
		Licenciatura en Física}\\[2cm]

\textsc{{\Large\bfseries Informe final}}\\
\textsc{\large Año de prácticas}\\[2cm]

\textsc{\large Supervisado por:\\
		\textbf{Dr. Carlos Pineda (IF-UNAM) y\\
		M.Sc. Juan Diego Chang (ICFM-USAC)}}


%------------------------------------------------                                                                                             
%       Date                                                                                                                                  
%------------------------------------------------                                                                                             
\vfill\vfill\vfill % Position the date 3/4 down the remaining page
\vfill\vfill\vfill

{\large xx de noviembre, 2020} % Date, change the \today to a set date if you want to be precise                                                              

%------------------------------------------------                                                                                             
%       Logo                                                                                                                                  
%------------------------------------------------                                                                                             

%\vfill\vfill                                                                                                                                 
%\includegraphics[width=0.2\textwidth]{placeholder.jpg}\\[1cm] % Include a department/university logo                                                                                                                                      

%----------------------------------------------------------------------------------------                                                     

\vfill % Push the date up 1/4 of the remaining page  
\end{titlepage}

\newtheorem{definition}{Definición}[section]

\newtheorem{teorema}{Teorema}[section]

\newtheorem{property}{Propiedad}[section]
% }}}

\chapter{Termodinámica}
\section{Trabajo}
\begin{enumerate}
\item Durante un proceso cuasiestático un sistema atraviesa en todo momento 
	  estados infinitesimalmente cercanos al equilibrio termodinámico y, 
	  por consiguiente, la dinámica de los estados puede ser descrita
	  en términos de coordenadas termodinámicas $\Rightarrow$ existe 
	  una ecuación de estado que describe estos estados 
\item Equilibrio termodinámico: eq. mecánico $+$ eq. térmico $+$ 
 	  eq. químico
\item Diferencial de trabajo de un sistema de un gas en un pistón:
\begin{equation}
\dbar W= PAdx = -PdV,
\end{equation}
el signo negativo asegura que en una compresión $(dV<0)$ $\dbar W>0$ 
(trabajo sobre el sistema, i.e. el sistema gana energía). 
\item El trabajo realizado sobre un sistema hidrostático depende de 
la trayectoria que se siga en el diagrama de estados (a diferencia
del trabajo realizado por la fuerza gravitacional, por ej). 
\end{enumerate}


%\bibliographystyle{abbrv}
%\bibliography{references}
\end{document}


