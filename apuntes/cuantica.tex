\chapter{Mecánica Cuántica}

\section{Herramientas elementales}
\begin{enumerate}
	\item \textbf{Ecuación de Schrödinger:}
	\begin{align*}
		i\hbar \pdv{\ket{\psi}}{t}=H\ket{\psi}
	\end{align*}
	
	\item Un conjunto
	se dice que es un conjunto completo de observables que conmutan si existe
	una única base ortonormal común de eigevectores.
	
	\item \textbf{Postulados:}
	\begin{itemize}
	\item Los estados de un sistema aislado se describen por vectores en 
	un espacio de Hilbert.
	\item Cada cantidad física tiene asociado un operador Hermítico que la describe
	y que actúa sobre el espacio de Hilbert del sistema. 
	\item El único posible valor de la medición de un observable son sus 
	eigenvalores. 
	\item Cuando un observable se mide sobre un sistema, la probabilidad
	de medir un eigenvalor u otro es igual al módulo cuadrado de la proyección
	del estado del sistema sobre el eigenvector correspondiente al eigenvalor que 
	resulta de la medición. Si el eigenvalor medido es degenerado, entonces 
	la probabilidad es igual a la suma del módulo cuadrado de las proyecciones 
	del estado del sistema sobre cada uno de los vectores de la base del 
	subespacio propio asociado al eigenvalor degenerado. 
	\item Luego de una medición, el sistema se encuentra en la proyección 
	del estado previo a la medición sobre el subespacio propio asociado 
	con el eigenvalor resultado de la medición. 
	\item La evolución del estado de un sistema está determinado 
	por la ecuación de Schrödinger. 
	\end{itemize}
	
	\item Dos observables se dice que no conmutan cuando sus operadores 
	Hermíticos asociados no conmutan, lo cual implica físicamente que no 
	se pueden medir simultáneamente. Dicho de otro modo, el orden de la 
	medición importa. 
	
	\item \textbf{Cuando el Hamiltoniano del sistema es independiente 
	del tiempo:} el estado del sistema para $t>t_0$ es 
	\begin{align}
	\ket{\psi(t)}=\sum_{n,i}\braket{\phi_n^i}{\psi(0)}e^{-iE_nt/\hbar}\ket{\phi_n^i}
	\end{align}
\end{enumerate}   