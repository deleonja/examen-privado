\chapter{Física General}
\section{Materia Condensada}
\begin{enumerate}
	\item \textbf{Cristal}: sólo en el que los átomos están ordenados 
	de tal forma que forman una red. Una red es un arreglo de puntos
	donde el mismo patrón se repite una y otra vez. 
	
	\item \textbf{Red de Bravais}: red en la que la vecindad de cada punto 
	es exactamente igual para todos los puntos en la red. En 2 dimensiones, 
	cada punto puede describirse con el vector 
	\begin{align}
		\vec{R}=n_1\vec{a}_1+n_2\vec{a}_2,
	\end{align}
	donde $\vec{a}_i$ son llamados los vectores primitivos.
	
	\item \textbf{Celda primitiva}: Celda unitaria que contiene a un único 
	punto de la red.
	
	\item \textbf{Zona de Wigner-Seitz}: celda primitiva que contiene toda 
	la información de la red, incluyendo sus simetrías. Se construye de la siguiente
	manera: se escoge un punto de la red como el origen y se trazan segmentos
	hacia cada uno de sus próximos vecinos, la zona de Wigner-Seitz se define 
	como el área que encierran todas las bisectrices perpendiculares de los 
	segmentos dibujados anteriormente. 
\end{enumerate}