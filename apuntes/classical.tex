\chapter{Mecánica Clásica}

\section{Formulación Lagrangiana}
Se define al lagrangiano $\L$ como 
\begin{align}
\L = T-U
\label{eq:lagrangian},
\end{align}
donde $T$ es la energía cinética y $U$ la energía potencia del sistema.
¿Por qué considerar a $\L$ si no es igual a la energía total?
Bueno, porque
\begin{align}
\pdv{\L}{x}=-\pdv{U}{x}&=F_x &
\pdv{\L}{\dot{x}}=\pdv{T}{\dot{x}}=m\dot{x	}&=p_x.
\end{align}
Donde hemos asumido que $U$ es función de las coordenadas y 
$T$ es función de las velocidades. Notemos que 
\begin{align}
\pdv{\L}{x}=\dv{t}\pdv{\L}{\dot{x}}.
\label{eq:lagrange-eq}
\end{align}
Si consideramos una partícula en 3 dimensiones las ecuaciones 
resultan exactamente análogas para cada una de las coordenadas 
rectangulares.

\textbf{Principio de Hamilton}: La trayectoria que sigue una partícula 
entre dos puntos 1 y 2 en un intervalo $t_1$ y $t_2$ es tal 
que la integral de acción 
\begin{align}
S=\int_{t_1}^{t_2} \L \ \d t,
\end{align}
es estacionaria cuando se integra a lo largo de la trayectoria que sigue 
la partícula. 

Tres cosas son equivalentes en este punto. La trayectoria de una partícula
está determinada por 
\begin{enumerate}
\item $\vb{F}=m\vb{a}$
\item El lagrangiano $\L$, mediante la ecuación \eqref{eq:lagrange-eq}
\item El principio de Hamilton
\end{enumerate}

\textbf{Coordenadas generalizadas:} coordenadas $q_i$ con la propiedad
de que cada posición $\vb{r}$ especifica a un único valor de $(q_i)$ y 
viceversa.

El lagrangiano es una función que depende de las coordenadas 
generalizadas y la primera derivada temporal de estas coordenadas. Es 
decir, 
\begin{align}
\L = \L \qty(q_i,\dot{q}_i).
\end{align}
Derivado de esto, se conoce como \textit{fuerza generalizada} a 
$\partial \L /\partial q_i$ y el momentum generalizado a 
$\partial \L /\partial \dot{q}_i$.

\textbf{Grados de libertad:} Si las coordenadas generalizadas de un 
sistema están restringidas, como el caso de un péndulo simple, entonces
el número de coordenadas generalizadas es menor a $3N$ donde $N$ es
el número de partículas en el sistema. A este número de coordenadas
generalizadas se le conoce como grados de libertad.

\begin{itemize}
\item \textbf{Coordenadas ignorables o ciclicas:} se dice que $q_i$ es 
una coordenada ignorable cuando el lagrangiano $\L$ es independiente
de $q_i$ (pero sí puede depender de $\dot{q}_i$). Cuando $q_i$ es 
una coordenada ignorable se dice que $\L$ es invariante cuando 
$q_i$ varía, lo que implica que el momentum generalizado $p_i$ se 
conserva.
\end{itemize}
