\chapter{Electromagnetismo}
\section{Electrostática}
\subsection{Campo eléctrico y ley de Gauss}
\begin{itemize}
\item El campo eléctrico producido por una densidad de carga $\rho(\vb{r})$
es
\begin{align}
\frac{q}{4\pi \epsilon_0}\int \frac{\rho(\vb{r}')}{r^2}\uv{r}dV'.
\end{align}

\item Se define el flujo de campo eléctrico $\vb{E}$ a través de 
una superficie $\mathcal{S}$ como
\begin{align}
\Phi_E\equiv\int _{\mathcal{S}}\vb{E}\vdot d\vb{a}.
\end{align}

\item \textbf{Ley de Gauss:}
\begin{align}
\oint \vb{E}\vdot d\vb{a}=\frac{1}{\ep}q_{\text{enc}},
\end{align}
el flujo de campo eléctrico atravesando una superficie cerrada
es proporcional a la carga que se encuentra encerrada dentro 
de la superficie. La dirección del flujo nos indica si tenemos un 
sumidero o una fuente y, al mismo tiempo, el signo de la carga
encerrada.

\item Es importante aprender a pasar de la forma integral, a la forma
diferencial de la ley de Gauss. Veamos a continuación. 
\begin{align*}
\oint_{\mathcal{S}} \vb{E}\vdot d\vb{a}=\int_{\mathcal{V}}
\qty(\div{\vb{E}})dV.
\end{align*}
\textit{\textbf{Paréntesis (teorema de la divergencia)}: la integral 
de superficie cerrada de un campo vectorial es igual a la integral 
de volumen de la divergencia del campo vectorial, integrado sobre
el volumen que encierra la superficie,}

Recordando que 
\begin{align*}
q_{\text{enc}}&=\int_{\mathcal{V}}\rho \ dV,
\end{align*}
por consiguiente, 
\begin{align*}
\int_{\mathcal{V}}\qty(\div{\vb{E}})dV=
\frac{1}{\ep}\int_{\mathcal{V}}\rho \ dV,
\end{align*}
puesto que la integración es sobre el mismo volumen, entonces
los integrandos deben ser iguales,
\begin{align}
\div{\vb{E}}=\frac{\rho}{\ep}.
\end{align}

La ley de Gauss es siempre cierta, \textit{pero no siempre útil}.
Particularmente, las simetrías son cruciales para poder aprovechar 
el resultado establecido en la ley de Gauss:
\begin{enumerate}
\item Simetría esférica: superficie gaussiana una esfera concéntrica.
\item Simetría cilíndrica: superficie gaussiana un cilindro coaxial. 
\item Simetría plana: superficie gaussiana una caja de pastillas.
\end{enumerate}
Aprovecharse de las simetrías $+$ utilizar el principio de superposición
hace que la ley de Gauss se vuelva especialmente útil para su aplicación.

\item $\curl{\vb{E}}=0$. Tomemos el caso de una carga puntual
$q$, cuyo campo eléctrico $E$ es
\begin{align}
\vb{E}=\frac{1}{4\pi\ep}\frac{q}{r^2}\uv{r}.
\end{align}
Ahora calculemos la integral de línea entre dos puntos arbitrarios
$a$ y $b$ del espacio, 
\begin{align*}
\int_a^b \vb{E}\vdot d\vb{l}.
\end{align*}
En coordenadas esféricas, $d\vb{l}=dr\uv{r}
+rd\theta \uv{\boldsymbol{\theta}}+
r\sin\theta d\phi \uv{\boldsymbol{\phi}}$, entonces
\begin{align*}
\int_a^b\vb{E}\vdot d\vb{l}&=\int_a^b\frac{1}{4\pi\ep}\frac{q}{r^2}dr\\
&=\frac{q}{4\pi\ep}\qty(\frac{1}{r_a}-\frac{1}{r_b}),
\end{align*}
por lo tanto, al integrar sobre una trayectoria cerrada la integral 
debe ser igual a cero:
\begin{align}
\oint\vb{E}\vdot d\vb{l}=0.
\end{align}
Por consiguiente, aplicando el teorema de Stokes:
\begin{align}
\oint\vb{E}\vdot d\vb{l}&=\int \qty(\curl{\vb{E}})\vdot d\vb{a}=0\\
\therefore \hspace{5mm} \curl{\vb{E}}&=0.
\end{align}
Este resultado se sigue satisfaciendo para cualquier distribución 
estática de cargas, pues si tenemos $N$ cargas, el campo eléctrico
de la distribución está determinado por 
\begin{align*}
\vb{E}=\vb{E}_1+\vb{E}_2+\ldots+\vb{E}_N\\
\end{align*}
\begin{align}
\therefore \hspace{5mm} 
\curl{\vb{E}}=\curl{\vb{E}_1}+\curl{\vb{E}_2}+\ldots+
\curl{\vb{E}_N}=0,
\end{align}
dado que cada rotor se debe hacer cero.
\end{itemize}
\subsection{Potencial eléctrico}
\begin{itemize}
\item El \textbf{potencial eléctrico} es una función escalar que se define
como
\begin{align}
V(\vb{r})\equiv -\int_{\mathcal{O}}^r\vb{E}\vdot d\vb{l},
\end{align}
con $\mathcal{O}$ un punto de referencia.

\item El teorema fundamental de los gradientes establece que
\begin{align*}
V(\vb{r}_b)-V(\vb{r}_a)=\int_a^b \grad V\vdot d\vb{l},
\end{align*}
por consiguiente, 
\begin{align*}
\int_a^b \grad V\vdot d\vb{l}&=-\int_a^b \vb{E}\vdot d\vb{l}\\
\vb{E}&=-\grad V.
\end{align*}

\item Es una consecuencia de $\curl{\vb{E}}=0$ que las componentes
$E_i$ de $\vb{E}$ no sean completamente independientes entre ellas.
De hecho, la formulación del potencial eléctrico explota al máximo 
esto.

\item \textbf{Ecuación de Poisson:}
\begin{align}
\grad^2V=\frac{\rho}{\ep}.
\end{align}

\item Cálculo del potencial eléctrico $V$ a partir de la distribución de
carga $\rho$:
\begin{align}
\frac{1}{4\pi\ep}\int\frac{\rho(\vb{r}')}{r}\ dV'.
\end{align}

\item \textbf{Discontinuidad al cruzar una superficie de carga $\sigma$}:
supongamos una superficie con densidad superficial de cargar $\sigma$.
Consideremos un área $A$ sobre la superficie, tan pequeña como 
sea necesario para que $\sigma=\text{cte.}$, y apliquemos 
la ley de Gauss con una caja de pastillas gaussiana. Los lados 
de la caja de pastillas (componente paralela a la superficie) no 
contribuyen nada al flujo neto de campo eléctrico, por lo tanto, 
sumando las contribuciones perpendiculares a la superficie se 
tiene $\qty(E_{\text{arriba}}-E_{\text{abajo}})A=\sigma A/\ep$,
entonces $E_{\text{arriba}}-E_{\text{abajo}}=\sigma /\ep$. Por 
lo cual, se concluye que existe una discontinuidad de la componente 
normal de $\vb{E}$ al atravesar una superficie de carga que es igual 
a $\sigma/\ep$. Las componentes paralelas a la superficie son continuas
y es consecuencia directa de 
\begin{align}
\oint \vb{E}\vdot d\vb{l}=0.
\end{align}
No obstante, el potencial eléctrico es continuo a través de cualquier
frontera, ya que 
\begin{align*}
V_{\text{arriba}}-V_{\text{abajo}}=-\int_a^b\vb{E}\vdot d\vb{l};
\end{align*}
conforme la trayectoria entre $a$ y $b$ se hace igual a cero, 
también lo hace el valor de la integral, por consiguiente
\begin{align}
V_{\text{arriba}}=V_{\text{abajo}}.
\end{align}
\end{itemize}

\subsection{Trabajo y energía}
\begin{itemize}
\item De lo que sabemos de mecánica el trabajo sobre un sistema 
para moverlo desde $a$ hasta $b$ es 
\begin{align}
W=\int_a^b\vb{F}\vdot d\vb{l}.
\end{align}
Por consiguiente, el trabajo para mover una carga $q$ es
\begin{align}
W=-q\int_a^b\vb{E}\vdot d\vb{l}=q\qty(V_b-V_a)
\end{align}

\item La energía de una configuración de cargas se define 
como la cantidad de trabajo necesario para armar la configuración:
\begin{align}
W=\frac{1}{2}\sum_i^nq_iV(\vb{r}_i),
\label{eq:ch3-W-point}
\end{align}
donde $V$ es el potencial generado por el resto de las cargas 
en la configuración.

\item Si en vez tenemos una distribución continua de carga $\rho$
entonces podemos reescribir la ecuación \eqref{eq:ch3-W-point}
como
\begin{align}
W=\frac{1}{2} \int \rho Vd\tau.
\end{align}
Si utilizamos la ley de Gauss podemos resscribir esta ecuación como
\begin{align}
W&=\frac{\ep}{2} \int \qty(\div{\vb{E}}) Vd\tau\nonumber\\
&=\frac{\ep}{2}\qty(-\int \vb{E}\vdot \qty(\grad V)d\tau
+\oint V\vb{E}\vdot d\vb{a}),
\end{align}
pero como $\grad V=-\vb{E}$,
\begin{align}
W&=\frac{\ep}{2}\qty(\int E^2d\tau+\oint V\vb{E}\vdot d\vb{a}),
\end{align}
consideramos la integración sobre todo el espacio, y de esa manera,
\begin{align}
W&=\frac{\ep}{2}\int E^2d\tau.
\end{align}
\end{itemize}

